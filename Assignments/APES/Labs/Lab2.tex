\PassOptionsToPackage{unicode=true}{hyperref} % options for packages loaded elsewhere
\PassOptionsToPackage{hyphens}{url}
%
\documentclass[]{article}
\usepackage{lmodern}
\usepackage{amssymb,amsmath}
\usepackage{ifxetex,ifluatex}
\usepackage{fixltx2e} % provides \textsubscript
\ifnum 0\ifxetex 1\fi\ifluatex 1\fi=0 % if pdftex
  \usepackage[T1]{fontenc}
  \usepackage[utf8]{inputenc}
  \usepackage{textcomp} % provides euro and other symbols
\else % if luatex or xelatex
  \usepackage{unicode-math}
  \defaultfontfeatures{Ligatures=TeX,Scale=MatchLowercase}
\fi
% use upquote if available, for straight quotes in verbatim environments
\IfFileExists{upquote.sty}{\usepackage{upquote}}{}
% use microtype if available
\IfFileExists{microtype.sty}{%
\usepackage[]{microtype}
\UseMicrotypeSet[protrusion]{basicmath} % disable protrusion for tt fonts
}{}
\IfFileExists{parskip.sty}{%
\usepackage{parskip}
}{% else
\setlength{\parindent}{0pt}
\setlength{\parskip}{6pt plus 2pt minus 1pt}
}
\usepackage{hyperref}
\hypersetup{
            pdfborder={0 0 0},
            breaklinks=true}
\urlstyle{same}  % don't use monospace font for urls
\setlength{\emergencystretch}{3em}  % prevent overfull lines
\providecommand{\tightlist}{%
  \setlength{\itemsep}{0pt}\setlength{\parskip}{0pt}}
\setcounter{secnumdepth}{0}
% Redefines (sub)paragraphs to behave more like sections
\ifx\paragraph\undefined\else
\let\oldparagraph\paragraph
\renewcommand{\paragraph}[1]{\oldparagraph{#1}\mbox{}}
\fi
\ifx\subparagraph\undefined\else
\let\oldsubparagraph\subparagraph
\renewcommand{\subparagraph}[1]{\oldsubparagraph{#1}\mbox{}}
\fi

% set default figure placement to htbp
\makeatletter
\def\fps@figure{htbp}
\makeatother


\date{}

\begin{document}

Keelan Krinsky: 1634953

\hypertarget{fundamentals-of-ecology-lab-2.}{%
\section{Fundamentals of ecology: Lab
2.}\label{fundamentals-of-ecology-lab-2.}}

\hypertarget{questions}{%
\subsection{Questions}\label{questions}}

\hypertarget{one}{%
\subsection{One}\label{one}}

Individual in the context fo a unitary organism implies that the
organism has a unique genetic make up. (distinct from other individual
in the same area/ of the same species)

Individual in the context of a modular organism implies some form of
physical separation between individual. Such individuals are usually
morphologically connected, and genetically identical but may still
display phenotypic plasticity-slightly differing phenotypes- depending
on their immediate physical surroundings .

\hypertarget{two}{%
\subsubsection{Two}\label{two}}

\hypertarget{annual}{%
\paragraph{Annual}\label{annual}}

A plant which has a life cycle which lasts over a period of less than a
year. this life cycle includes:

\begin{enumerate}
\def\labelenumi{\arabic{enumi}.}
\tightlist
\item
  Germination
\item
  Juvenile phases(seedlings)
\item
  An adult phase (in which the organism is capable of reproduction)
\item
  Senescent/ dying phase.
\end{enumerate}

these plants tend to be herbaceous, and relatively small.

\hypertarget{perennial}{%
\paragraph{Perennial}\label{perennial}}

Plants which have a life cycle which extend over many year. the
essential stages of their life style however remain the same as those of
annual plants.

These plants tend to be large and woody.

\hypertarget{semel--parous}{%
\paragraph{Semel- parous}\label{semel--parous}}

Semel-parous individuals have a long juvenile phase, which takes up the
majority of their life, followed by a single reproductive event just
before their death.

\hypertarget{iteroparous}{%
\paragraph{Iteroparous}\label{iteroparous}}

Iteroparous individuals can reproduce many times/ continuously through
their adult life before senescence sets in.

\hypertarget{examples}{%
\subparagraph{Examples}\label{examples}}

Annual Semel-parous.

PLANT: \emph{Amaranthus cruentus}

ANIMAL: \emph{Dolania Americana} (A mayfly species)

Annual Iteroparous

PLANT:-------

ANIMAL:\emph{Anopheles albimanus} (Mosquito Spp)

\hypertarget{perennial-semel-parous.}{%
\subparagraph{Perennial Semel-parous.}\label{perennial-semel-parous.}}

PLANT: \emph{Lobelia telekii} (Centuary Plant)

ANIMAL: \emph{Sepioteuthis sepioidea} (Caribbean reef squid)

\hypertarget{perennial-iteroparous.}{%
\paragraph{Perennial Iteroparous.}\label{perennial-iteroparous.}}

PLANT: \emph{Quercus alba} (White Oak)

ANIMAL \emph{Tragelaphus strepsiceros} (Kudu spp)

Annual Iteroparous plant species proved very difficult to find,
(although no source seemed to specifically negate their existence).
Perhaps the reason why so few/ if any annual plant species are
Iteroparous is because the resource necessary to fuel reproductive
activity take any given plant too long to accrue for two or more
reproductive events to feasibly occur in one season.

\hypertarget{three}{%
\subsubsection{Three}\label{three}}

\hypertarget{aggregated}{%
\paragraph{Aggregated}\label{aggregated}}

Individuals are not evenly distributed but rather grouped into areas of
higher density, with groups separated by (normally large) areas of lower
density.

This distribution may result from a scattered resource distribution,
with high density areas surrounding high concentrations/amounts of these
resources. It could also be caused by social factors which lead to
aggregation into large herds, as the predominant social structure.

\hypertarget{random}{%
\paragraph{Random}\label{random}}

In a random distribution individuals are evenly distributed with a
(relatively) constant population density throughout.

A random distribution will occur almost as a default distribution if
there is no underlying pattern of resource scarcity, or inter and intra
species interactions which could cause emergent patterns/ properties in
the population density distribution.

\hypertarget{regular}{%
\paragraph{Regular}\label{regular}}

In a Regular distribution there is a set patten/ organisation to the
distribution of individuals, (usually as an emergent property of the
movement patterns of the individual themselves) which exact pattern
emerges will depend on the specific population under consideration.

regular arrangements may be as a result of intra species social
structures such as territorial behaviour. Alternatively it may result
from interspecies interactions such as the regular arrangement of plant
in Anthromes for agriculture, or landscaping.

\hypertarget{four}{%
\subsubsection{Four}\label{four}}

Because most of South Africa in terms of total surface area,is virtually
uninhabited (by humans). On a day to day, most South Africans live in
high density cosmopolitan settlements, or at least medium density rural
settlements, concentrated into a very small percentage of the total
available area.

A similar contrast may appear in other species as a result of aggregate
population distribution, or if a given population was in the process of
migrating into a new much larger territory.

Such a Contrast may not be seen in a population which has reached the
carrying capacity from a given area, and is randomly distributed in that
area. Furthermore population with regular distribution, depending on the
specific distribution pattern may not show this contrast.

\hypertarget{five}{%
\subsubsection{Five}\label{five}}

For unitary organisms genetic differences between individuals can lead
to a degree of differentiation. furthermore as individuals are not
physically connected intraspecific competition will lead to a certain
maximum population density.

For modular organisms, although individuals may compete to an extent for
resources, detriments to one individuals may, by extension, detriment
other individuals in the network which were deriving some resources from
the individual in question. As a result competition is commonly limited
by hormonal activity between individuals in the network. to prevent
individual from negatively impacting on each others growth. Phenotypic
plasticity can also reduce competition, both at the individual and
population level, although unlike with unitary organisms it is unlikely
to lead ultimately to speciation.

\hypertarget{six}{%
\subsection{Six}\label{six}}

\begin{enumerate}
\def\labelenumi{\arabic{enumi}.}
\item
  Birth Rate (Auxiliary variable)
\item
  Births (Auxiliary variable)
\item
  ImmigrationRate (Auxiliary variable)
\item
  Immigration (Auxiliary variable)
\item
  DeathRate (Auxiliary variable)
\item
  Deaths. (Auxiliary variable)
\item
  EmigrationRate (Auxiliary variable)
\item
  Emigration (Auxiliary variable)

  Auxiliary variable are used in all these cases as these variable are
  important in calculating the final value of the main stock, but are
  not of them selves the main focus of the model.
\item
  Number of elephants. (Stock)
\end{enumerate}

A stock is used as this is the main amount/variable under consideration,
and the level of this variable over time is an important prediction of
the model.

There are also connectors, to link interconnected variable and flows to
track the overall movement of elephants into and out of the population.

\hypertarget{seven}{%
\subsection{Seven}\label{seven}}

The direct output of the model is merely a data set describing the value
of the stock ``NoElephants'' over time, this data shows a rapid,
exponential type growth curve, with no deviation or oscillation over
time.\\
The temporal scale of the model has an extent of a hundred years, and a
resolution of 1 year.

The resolution seems appropriate considering the biology to be modelled
as although an individual elephant will not produce offspring each year,
(gestation period alone is closer to 2 years and time between successive
pregnancies is also in the range of two to four years), in a reasonable
sized population there will still be births each year. Similarly,
although elephants will normally reproduce seasonally, that is they will
not give birth in winter, a scale of 6 months (to follow the main
seasons) would not be seasonable as births will still only occur once a
year.

The extent of the time scale however seems too short to give a
reasonable overview of population size overtime. This is because
elephants are relatively long lived organisms 50-70 years, so 100 years
is not long enough to allow for distinct generations to emerge, and to
see the long term response in population size to external factors. There
are also other factors influencing population size, such as migratory
patterns, or resource distributions which may change on larger time
scales which also will not be predicted by the model.

\hypertarget{eight}{%
\paragraph{Eight}\label{eight}}

\begin{enumerate}
\def\labelenumi{\arabic{enumi}.}
\item
  Births (Birthrate x NoElephants)
\item
  Deaths (Deathrate x NoElephants)
\item
  Immigration (Immigrationrate x NoElephants)
\item
  Emigration (Emigrationrate x NoElephants)
\end{enumerate}

this information tells Vensim the relative number of
Births,death,immigrations, and emigrations per year (or how to calculate
this value). It could also be though of as telling vensim what numbers
to add to/ subtract from the stock NoElephants.

\hypertarget{nine}{%
\subsection{Nine}\label{nine}}

Modifying the values in any of the rate coefficients will lead to a
decrease in population growth if the DeathRate or Emigration rate are
increased, and a increase in population growth rate if the BirthRate of
ImmigrationRate are increased.

the effect of the change will depend purely of the extent of the numeric
change to the constant. For example increasing Birthrate or Immigration
rate by the same numeric amount will have the same impact on the overall
population. changes which lead to an increase in population growth rate
will have more overall effect, as they will increase the exponential
growth, which leads to increasingly large changes in actual population
size. conversely changes which decrease the population growth rate, push
population growth further towards exponential decay, where population
size decreased by by less and less each year.

\hypertarget{ten}{%
\subsection{Ten}\label{ten}}

the population exhibits exponential growth. this type of growth may be
realistic when a population is expanding into a new territory where
there is no, or very limited resource scarcity, particularly if there
are no/few natural predictors or parasites for that population in the
new environment. In these circumstances there would be no real limits to
growth, and increased population sizes would not slow down the
population growth. The same might be true after a catastrophic event
destroyed large sections of the population leaving only a few
individuals to expand back to the original population size (although
this assumes that the catastrophic event did not detrimentally effect
the ) resource base of the population)

NOTE: Even in the above mentioned cases however this growth model is
still not entirely realistic as the is no natural variation in
population size which would normally result from accidental deaths,
variations in birth rate, variation in life expectancy and a myriad of
other contributive factors.

\end{document}
