\PassOptionsToPackage{unicode=true}{hyperref} % options for packages loaded elsewhere
\PassOptionsToPackage{hyphens}{url}
%
\documentclass[]{article}
\usepackage{lmodern}
\usepackage{amssymb,amsmath}
\usepackage{ifxetex,ifluatex}
\usepackage{fixltx2e} % provides \textsubscript
\ifnum 0\ifxetex 1\fi\ifluatex 1\fi=0 % if pdftex
  \usepackage[T1]{fontenc}
  \usepackage[utf8]{inputenc}
  \usepackage{textcomp} % provides euro and other symbols
\else % if luatex or xelatex
  \usepackage{unicode-math}
  \defaultfontfeatures{Ligatures=TeX,Scale=MatchLowercase}
\fi
% use upquote if available, for straight quotes in verbatim environments
\IfFileExists{upquote.sty}{\usepackage{upquote}}{}
% use microtype if available
\IfFileExists{microtype.sty}{%
\usepackage[]{microtype}
\UseMicrotypeSet[protrusion]{basicmath} % disable protrusion for tt fonts
}{}
\IfFileExists{parskip.sty}{%
\usepackage{parskip}
}{% else
\setlength{\parindent}{0pt}
\setlength{\parskip}{6pt plus 2pt minus 1pt}
}
\usepackage{hyperref}
\hypersetup{
            pdfborder={0 0 0},
            breaklinks=true}
\urlstyle{same}  % don't use monospace font for urls
\setlength{\emergencystretch}{3em}  % prevent overfull lines
\providecommand{\tightlist}{%
  \setlength{\itemsep}{0pt}\setlength{\parskip}{0pt}}
\setcounter{secnumdepth}{0}
% Redefines (sub)paragraphs to behave more like sections
\ifx\paragraph\undefined\else
\let\oldparagraph\paragraph
\renewcommand{\paragraph}[1]{\oldparagraph{#1}\mbox{}}
\fi
\ifx\subparagraph\undefined\else
\let\oldsubparagraph\subparagraph
\renewcommand{\subparagraph}[1]{\oldsubparagraph{#1}\mbox{}}
\fi

% set default figure placement to htbp
\makeatletter
\def\fps@figure{htbp}
\makeatother


\date{}

\begin{document}

\hypertarget{step-one}{%
\paragraph{Step one}\label{step-one}}

\(42mg/100ml=4.2 \quad x \quad 10^{-4}g/ml\)

\(\frac{ 4.2 \quad x \quad 10^{-4}g/ml}{4kcal/g}=1.05 \quad x \quad 10^{-4}kcal/ml\)

\hypertarget{step-two}{%
\paragraph{Step two}\label{step-two}}

as the mixture was diluted by a factor of 10 to get to the new solution,
the concentration must be multiplied by a factor of ten to get back to
the original concentration.
\(1.05 \quad x \quad 10^{-4}kcal/ml \quad x \quad 10 =1.05 \quad x \quad 10^{-3}kcal/ml\)

\hypertarget{step-three}{%
\paragraph{Step three}\label{step-three}}

now we know that this 2.5ml sample which we now have the concentration
of was taken from a 10.5 ml solution (10ml from the drink and 0.5 ml of
HCl solution) but we need to get back to the concentration of the 10ml
sample so we do the same process again. if the dilution factor was
\(\frac{10.5}{10}\) we must now multiply by this factor to get the
original concentration:

\(10 =1.05 \quad x \quad 10^{-3}kcal/ml \quad x \quad \frac{10.5}{10}=1.1025 \quad x \quad 10^{-3}kcal/ml \quad\)

\hypertarget{step-four}{%
\paragraph{Step Four}\label{step-four}}

now we have the concentration of the 10ml sample and hence the
concentration of the original drink. as we are told a glass is 250 ml

\(250ml \quad x \quad 1.1025 \quad x \quad 10^{-3}=0.276kcal= 276cal\)

\end{document}
