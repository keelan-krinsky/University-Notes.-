\PassOptionsToPackage{unicode=true}{hyperref} % options for packages loaded elsewhere
\PassOptionsToPackage{hyphens}{url}
%
\documentclass[]{article}
\usepackage{lmodern}
\usepackage{amssymb,amsmath}
\usepackage{ifxetex,ifluatex}
\usepackage{fixltx2e} % provides \textsubscript
\ifnum 0\ifxetex 1\fi\ifluatex 1\fi=0 % if pdftex
  \usepackage[T1]{fontenc}
  \usepackage[utf8]{inputenc}
  \usepackage{textcomp} % provides euro and other symbols
\else % if luatex or xelatex
  \usepackage{unicode-math}
  \defaultfontfeatures{Ligatures=TeX,Scale=MatchLowercase}
\fi
% use upquote if available, for straight quotes in verbatim environments
\IfFileExists{upquote.sty}{\usepackage{upquote}}{}
% use microtype if available
\IfFileExists{microtype.sty}{%
\usepackage[]{microtype}
\UseMicrotypeSet[protrusion]{basicmath} % disable protrusion for tt fonts
}{}
\IfFileExists{parskip.sty}{%
\usepackage{parskip}
}{% else
\setlength{\parindent}{0pt}
\setlength{\parskip}{6pt plus 2pt minus 1pt}
}
\usepackage{hyperref}
\hypersetup{
            pdfborder={0 0 0},
            breaklinks=true}
\urlstyle{same}  % don't use monospace font for urls
\setlength{\emergencystretch}{3em}  % prevent overfull lines
\providecommand{\tightlist}{%
  \setlength{\itemsep}{0pt}\setlength{\parskip}{0pt}}
\setcounter{secnumdepth}{0}
% Redefines (sub)paragraphs to behave more like sections
\ifx\paragraph\undefined\else
\let\oldparagraph\paragraph
\renewcommand{\paragraph}[1]{\oldparagraph{#1}\mbox{}}
\fi
\ifx\subparagraph\undefined\else
\let\oldsubparagraph\subparagraph
\renewcommand{\subparagraph}[1]{\oldsubparagraph{#1}\mbox{}}
\fi

% set default figure placement to htbp
\makeatletter
\def\fps@figure{htbp}
\makeatother


\date{}

\begin{document}

\hypertarget{basic-notes-on-course}{%
\section{Basic notes on course}\label{basic-notes-on-course}}

The course consists of two parts

\hypertarget{descriptive}{%
\subsection{Descriptive}\label{descriptive}}

First block Large data set, few summary measures, extract essential info
from data

\hypertarget{inferential}{%
\subsection{Inferential}\label{inferential}}

Second block

\hypertarget{definitions}{%
\section{Definitions}\label{definitions}}

\hypertarget{statistic}{%
\subsubsection{Statistic}\label{statistic}}

Value calculated from a set of sample data (latin letters used )
Descriptive summary measure for a sample

\hypertarget{parameter}{%
\subsubsection{Parameter}\label{parameter}}

Value calculated from data about the entire population (greek letters
are used ) Numeric description of characteristic of the population

\hypertarget{population}{%
\subsubsection{Population}\label{population}}

Set of all items under investigation

\hypertarget{target-population}{%
\subsubsection{Target population}\label{target-population}}

The entirety of the population of units which falls under the scope of
the study/ population about which inferences are to be made.

\hypertarget{sampled-population}{%
\subsubsection{Sampled population}\label{sampled-population}}

The section of the target population for which information is (can be)
collected.

\hypertarget{raw-data}{%
\subsubsection{Raw data}\label{raw-data}}

Data which is unorganised numerically and remains in the form in which
it was collected.

\hypertarget{array}{%
\subsubsection{Array}\label{array}}

An arrangement of raw data in de/ascending order of magnitude.

\hypertarget{randomness}{%
\subsubsection{Randomness}\label{randomness}}

Equal chance of selecting any member of the sample space.

\hypertarget{range}{%
\subsubsection{Range}\label{range}}

the difference between the largest value (observation) and the smallest
value (observation) of the data set

\hypertarget{cluster}{%
\subsubsection{Cluster}\label{cluster}}

a group of points that fall very closer together

\hypertarget{outlier}{%
\subsubsection{Outlier}\label{outlier}}

an observation of datum that is unusually far from the bulk of the data.

\hypertarget{relative-frequency}{%
\subsubsection{Relative Frequency}\label{relative-frequency}}

Proportion of the data having a certain property. Relative frequencies
give the proportion of the observations falling in a particular group.

\hypertarget{exam-notes}{%
\section{Exam notes}\label{exam-notes}}

Exam is representative of the entire course.

\hypertarget{general-background}{%
\section{General background}\label{general-background}}

\hypertarget{summary-of-statistic-process}{%
\subsection{Summary of statistic
process}\label{summary-of-statistic-process}}

\hypertarget{general}{%
\subsubsection{General}\label{general}}

Collect data, organise, summarize, present (graphs and shit), analyse,
interpret/valid conclusions, make decisions

\hypertarget{collection}{%
\subsubsection{Collection}\label{collection}}

Collect data from a set of unit which are under identical conditions,
but which nevertheless display natural random variation .

\hypertarget{sampling}{%
\subsubsection{Sampling}\label{sampling}}

A sample is taken if the whole measuring the whole population is
unfeasible. A sample should be unbiased/representative, ie it should
have all of the same conditions as the population as a whole.

\hypertarget{data}{%
\subsubsection{Data}\label{data}}

\hypertarget{quality}{%
\paragraph{Quality}\label{quality}}

\hypertarget{scales-of-measurement}{%
\subsubsection{Scales of measurement}\label{scales-of-measurement}}

All data can be sorted into one of the four scales of measurement,and
appropriate methods can validly be performed on it to give a meaningful
result.

\hypertarget{qualitative}{%
\paragraph{Qualitative}\label{qualitative}}

Relating to separate categories.

\hypertarget{nominal-scale}{%
\subparagraph{Nominal Scale}\label{nominal-scale}}

Categorical

Examples

\begin{itemize}
\tightlist
\item
  gender type
\item
  Colors.
\end{itemize}

NOTE: don't confuse the fact that the categories cannot be ordered with
the fact that data points within a given category can be ordered.

\hypertarget{ordinal}{%
\subparagraph{Ordinal}\label{ordinal}}

categorical data that can be ordered/ ranked.

Examples

\begin{itemize}
\tightlist
\item
  Grades
\end{itemize}

\hypertarget{quantitative.}{%
\paragraph{Quantitative.}\label{quantitative.}}

\hypertarget{discrete}{%
\subparagraph{Discrete}\label{discrete}}

Countable

Examples

\begin{itemize}
\tightlist
\item
  humans
\item
  atoms
\item
  (positive)integer numbers numbers.
\end{itemize}

\hypertarget{continuous}{%
\subparagraph{Continuous}\label{continuous}}

non- countable, smooth variation

examples

\begin{itemize}
\tightlist
\item
  time
\item
  space
\item
  (positive) real numbers.
\end{itemize}

\hypertarget{interval-scale.}{%
\subparagraph{Interval Scale.}\label{interval-scale.}}

difference between scale points is the same anywhere along the scale,
however the zero of the scale is not the true zero (complete absence) of
the variable which it measure.

Examples

\begin{itemize}
\tightlist
\item
  Fahrenheit scale
\end{itemize}

NOTE: does this scale include discrete data?

\hypertarget{ratio-scale}{%
\subparagraph{Ratio Scale}\label{ratio-scale}}

both differences between scale points and ratio's of data values are
meaningful, the scale has a true zero.

\hypertarget{selection-method}{%
\paragraph{Selection method}\label{selection-method}}

\hypertarget{direct-observation}{%
\subparagraph{Direct observation}\label{direct-observation}}

\hypertarget{interval-methods}{%
\subparagraph{interval methods}\label{interval-methods}}

\hypertarget{experimentation}{%
\subparagraph{Experimentation}\label{experimentation}}

Data is generate though the manipulation of variables under controlled
conditions data on the primary variable under study can be monitored and
recorded while conscious efforts are made by the researches to control
the effects of a number of influencing factors.

Examples

this method is usually used in Biology, Chemistry and other natural
sciences

\hypertarget{graphical-methods}{%
\section{Graphical Methods}\label{graphical-methods}}

\hypertarget{background}{%
\subsection{Background}\label{background}}

graphical techniques are often used to convey statical information
(results) to the public who may wish to use them.

\hypertarget{advantages}{%
\subsubsection{advantages}\label{advantages}}

\begin{itemize}
\tightlist
\item
  quick and easy to see what is going on
\item
  easy to pick out trends
\end{itemize}

\hypertarget{definitions-1}{%
\subsubsection{Definitions}\label{definitions-1}}

\hypertarget{range-1}{%
\paragraph{Range}\label{range-1}}

the difference between the largest value (observation) and the smallest
value (observation) of the data set

\hypertarget{cluster-1}{%
\paragraph{Cluster}\label{cluster-1}}

a group of points that fall very closer together

\hypertarget{outlier-1}{%
\paragraph{Outlier}\label{outlier-1}}

an observation of datum that is unusually far from the bulk of the data.

\hypertarget{relative-frequency-1}{%
\paragraph{Relative Frequency}\label{relative-frequency-1}}

Proportion of the data having a certain property. Relative frequencies
give the proportion of the observations falling in a particular group.

\hypertarget{types-of-graphs}{%
\subsection{Types of graphs}\label{types-of-graphs}}

\begin{itemize}
\tightlist
\item
  Dot Plots
\item
  Pie charts
\item
  Line graphs
\item
  Bar graphs.charts I stem and leaf plots
\item
  Histograms
\item
  Boxplots.
\end{itemize}

\hypertarget{dot-plot}{%
\subsubsection{Dot Plot}\label{dot-plot}}

A plot of values on a vertical or horizontal scale. each number is
represented by a dot and the dots must be distinct.

\hypertarget{pie-chart}{%
\subsubsection{Pie Chart}\label{pie-chart}}

A circle divided into segments. the size of each segment is proportional
to the importance of the data category of random variable relative to
the whole. (normally used for categorical data)

\hypertarget{line-graph}{%
\subsubsection{Line Graph}\label{line-graph}}

shows the value of one variable against another (usually time), used to
investigate trends between two variables.

\hypertarget{bar-chart}{%
\subsubsection{Bar Chart}\label{bar-chart}}

Presentation of the categorical or discrete data by means of bars/
blocks orientated side by side.

\hypertarget{stem-and-leaf-plot}{%
\subsubsection{Stem and Leaf plot}\label{stem-and-leaf-plot}}

numbers are split into two parts. A group of digits from the left of the
number (the leaf) and the remainder of the data (the stem) leaves and
stem are then arranged on either side of a vertical spacer.(\textbar{})

The graph requires a key to indicate where decimal point falls.

Good for plotting discrete of continuous data with many observations
(abut 15-300)

\hypertarget{histogram}{%
\subsubsection{Histogram}\label{histogram}}

A histogram is constructed form a frequency table, with bar continuous
on one another. The intervals are shown on the x axis, and the number of
items/data points within each interval is shown if the u Y axis.

A histogram is usually used for continuous variable but it can also be
used to plot discrete variables. If it is used on a discrete variable
then the variable is/ must be converted into a continuous variable.

\#\#\#\#Advantages Good for showing symmetry of data.

\hypertarget{r-coding}{%
\section{R coding}\label{r-coding}}

\hypertarget{background-1}{%
\subsection{Background}\label{background-1}}

Knowledge of R code will not be tested directly in the course, however
examples of data put through r code will be given and must be
interpreted.

\end{document}
