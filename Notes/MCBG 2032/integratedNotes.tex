\PassOptionsToPackage{unicode=true}{hyperref} % options for packages loaded elsewhere
\PassOptionsToPackage{hyphens}{url}
%
\documentclass[]{article}
\usepackage{lmodern}
\usepackage{amssymb,amsmath}
\usepackage{ifxetex,ifluatex}
\usepackage{fixltx2e} % provides \textsubscript
\ifnum 0\ifxetex 1\fi\ifluatex 1\fi=0 % if pdftex
  \usepackage[T1]{fontenc}
  \usepackage[utf8]{inputenc}
  \usepackage{textcomp} % provides euro and other symbols
\else % if luatex or xelatex
  \usepackage{unicode-math}
  \defaultfontfeatures{Ligatures=TeX,Scale=MatchLowercase}
\fi
% use upquote if available, for straight quotes in verbatim environments
\IfFileExists{upquote.sty}{\usepackage{upquote}}{}
% use microtype if available
\IfFileExists{microtype.sty}{%
\usepackage[]{microtype}
\UseMicrotypeSet[protrusion]{basicmath} % disable protrusion for tt fonts
}{}
\IfFileExists{parskip.sty}{%
\usepackage{parskip}
}{% else
\setlength{\parindent}{0pt}
\setlength{\parskip}{6pt plus 2pt minus 1pt}
}
\usepackage{hyperref}
\hypersetup{
            pdfborder={0 0 0},
            breaklinks=true}
\urlstyle{same}  % don't use monospace font for urls
\setlength{\emergencystretch}{3em}  % prevent overfull lines
\providecommand{\tightlist}{%
  \setlength{\itemsep}{0pt}\setlength{\parskip}{0pt}}
\setcounter{secnumdepth}{0}
% Redefines (sub)paragraphs to behave more like sections
\ifx\paragraph\undefined\else
\let\oldparagraph\paragraph
\renewcommand{\paragraph}[1]{\oldparagraph{#1}\mbox{}}
\fi
\ifx\subparagraph\undefined\else
\let\oldsubparagraph\subparagraph
\renewcommand{\subparagraph}[1]{\oldsubparagraph{#1}\mbox{}}
\fi

% set default figure placement to htbp
\makeatletter
\def\fps@figure{htbp}
\makeatother


\date{}

\begin{document}

\hypertarget{mcbg-2032-concepts}{%
\section{MCBG 2032 Concepts}\label{mcbg-2032-concepts}}

\hypertarget{genetics}{%
\subsection{Genetics}\label{genetics}}

\hypertarget{course-outline}{%
\subsubsection{Course outline}\label{course-outline}}

\begin{enumerate}
\def\labelenumi{\arabic{enumi})}
\tightlist
\item
  extension of Mendelian genetics
\item
  Chromosomes structure and changes
\item
  Vertebrate development
\end{enumerate}

\hypertarget{interactions-between-genes-alleles}{%
\subsection{Interactions between genes/
Alleles}\label{interactions-between-genes-alleles}}

\hypertarget{incomplete-dominance}{%
\subsubsection{Incomplete dominance}\label{incomplete-dominance}}

\hypertarget{genetic-molecular-cause}{%
\paragraph{Genetic/ molecular cause}\label{genetic-molecular-cause}}

The recessive allele does not produce any functioning protein. The
dominant allele does function protein but not in high enough quantities
to result in the same phenotypic effect as a heterozygous dominant

\hypertarget{phenotypic-effect}{%
\paragraph{phenotypic effect}\label{phenotypic-effect}}

A heterozygous individual will display a phenotype intermediate to that
of homozygous dominant and homozygous recessive

\hypertarget{dominance}{%
\subsubsection{Dominance}\label{dominance}}

\hypertarget{geneticmolecular-cause}{%
\paragraph{Genetic/molecular cause}\label{geneticmolecular-cause}}

The recessive allele does not produce any functioning protein. The
dominant allele does function protein, which it produces in comparative
amount to a homozygous dominant individual

\hypertarget{phenotypic-consequence.}{%
\paragraph{Phenotypic consequence.}\label{phenotypic-consequence.}}

the heterozygous phenotype is identical to that of the heterozygous
dominant individual.

\hypertarget{co-dominance}{%
\subsubsection{Co-dominance}\label{co-dominance}}

\hypertarget{genetic-molecular-cause-1}{%
\paragraph{Genetic/ molecular cause}\label{genetic-molecular-cause-1}}

Both alleles are transcribed/translated to form a functioning protein

\hypertarget{phenotypic-effect-1}{%
\paragraph{phenotypic effect}\label{phenotypic-effect-1}}

The effect comes from the full expression of both alleles

\hypertarget{dominance-hierarchies.}{%
\subsubsection{Dominance hierarchies.}\label{dominance-hierarchies.}}

It is often the case that there will be a number of alleles for one
gene. In which case a dominance hierarchy is usually present such that
allele A is Dominant to B is dominant to C and so forth (ie dominance in
a hierarchy is transitive)

\hypertarget{pleiotropy-check-spelling}{%
\subsubsection{pleiotropy (Check
spelling)}\label{pleiotropy-check-spelling}}

Genetic/molecular cause One gene locus affects the expression of several
characteristics. Phenotypic cause Phenotype of one characteristic will
change in conjunction with the phenotype of another characteristic

\hypertarget{notes-about-colour}{%
\subsubsection{Notes about colour}\label{notes-about-colour}}

As many genetic examples are based on colour it is worth understanding
the basics of colouring in animals. colour results from pigments which
are proteins, individual genes can either code directly for the
production of these proteins or as an intermediate/enzyme which aids in
the production of these proteins.

\hypertarget{white}{%
\paragraph{White}\label{white}}

white does not result from a specific pigment but rather from a lack of
pigment.

\hypertarget{co-dominance-and-incomplete-dominance}{%
\paragraph{Co-dominance and Incomplete
dominance}\label{co-dominance-and-incomplete-dominance}}

a co-dominant colour combination can look like an intermediate if it is
the combination of two pigments, however in the case where one of the
phenotypes is white then a intermediate can only be a result of
incomplete dominance

\hypertarget{important-examples}{%
\subsubsection{Important examples}\label{important-examples}}

\hypertarget{frog-colour}{%
\paragraph{Frog colour}\label{frog-colour}}

\hypertarget{details}{%
\subparagraph{details}\label{details}}

the skin colour of a certain from is controlled by one allele. There are
three different phenotypes observed, these are red, blue, and purple.

\hypertarget{explanation}{%
\subparagraph{explanation}\label{explanation}}

This is an example of co-dominance as the purple colour comes from a
combination of the red and blue pigments. that is both red and blue
pigments have been produced. We can conclude that the purple from is
heterozygous, the blue frog is homozygous blue and the red frog is
homozygous red.

\hypertarget{fruit-flies-eye-colour.}{%
\paragraph{Fruit flies eye colour.}\label{fruit-flies-eye-colour.}}

\hypertarget{details.}{%
\subparagraph{details.}\label{details.}}

the colour of fruit flies eyes is determined by two gene loci. a fly
with dominant alleles at the red locus will be red eyed, while a fly
with dominant alleles at the brown locus will be brown eyed. However a
fly with both dominant brown and cinnabar will be white eyed.

\hypertarget{explanation-1}{%
\subparagraph{explanation}\label{explanation-1}}

the white eyed phenotype is due to a form of inhibition similar to
epistasis stops either of the pigments form being produced if both
pigment genes are present.

\hypertarget{rabbit-skin-colour}{%
\paragraph{Rabbit Skin Colour}\label{rabbit-skin-colour}}

\hypertarget{details-1}{%
\subparagraph{details}\label{details-1}}

the skin colour of rabbits is determined by one gene locus, which has
four possible Alleles. All alleles code for the enzyme tyrosinase which
acts to convert tyrosine into dopamine a critical first set in the
production of hair pigment.

\#1 non functional

The most recessive Allele codes for a nonfunctioning version of
tyrosinase leading to albino rabbit.

\#2 heat sensitive

The next most recessive allele codes for a heat inactivated version of
tyrosine meaning that the final colour of the rabbits fur will depend on
the temperature conditions it was raised in, and also that within a
certain temperature range there will be a distinct colour difference
between the warmer and colder parts of the animal.

\#3 localised functionality

The next most recessive allele codes for a version of tyrosinase does
not fold quite ``correctly'' and as such does not glycosalate in the
same way as the dominant version. melanocytes which produce the grey
pigment will still accept this tyrosinase into their cytoplasm and as
such grey pigment will be produced. however as intracellular transport
is affected the tyrosinase either cannot reach or cannot enter
melanocytes capable of producing the brown pigments and so they rabbits
are full grey

\#4 fully functional

the most dominant allele codes for a fully functional form of tyrosinase
and the combination of grey and orange pigments produced gives the
rabbits a full brown coat.

\hypertarget{human-blood-types.}{%
\paragraph{human Blood types.}\label{human-blood-types.}}

\hypertarget{details-2}{%
\paragraph{details}\label{details-2}}

human blood types are determined by one gene locus with three alleles.
allele O is recessive, and codes for no functioning protein. allele A
and B are co dominant and each code for different sugars which are
attached to the outside of red blood cells.

\hypertarget{bombay-phenotype}{%
\paragraph{Bombay phenotype}\label{bombay-phenotype}}

\hypertarget{details-3}{%
\subparagraph{details}\label{details-3}}

parents A and AB, child O, mates with A, results B, AB and O.

\hypertarget{explanation-2}{%
\subparagraph{explanation}\label{explanation-2}}

A and B sugars are attached to red blood cells by appending them onto
the end of glycoproteins called Substance H which are already attached
to the cell. If a individual is homozygous recessive for substance H
genes, none will be produced and so even if the have A or B alleles they
will have a functionally O phenotype.

\#\#\#Alleles and genes A gene is a collection/ group of alleles which
could occur the same locus and which are associated with the same
traits.

\#\#\#lethal alleles alleles which are incompatible with the organisms
continued survival

\hypertarget{dominance-1}{%
\paragraph{dominance}\label{dominance-1}}

most lethal alleles are recessive because dominant lethal alleles
produce non functioning proteins only so any individual with even one of
the dominant allele would be unlikely to survive to reproductive age,
and so the Allele would very quickly be bread out of the population.

\hypertarget{sex-chromosomes}{%
\paragraph{sex chromosomes}\label{sex-chromosomes}}

lethal alleles, or alleles linked with low fertility located on the Y
chromosome in humans are very rare. disease related to homozygous XX are
more likely to be suffered by woman, as they will have one working copy
of the given gene so they can survive but one defective copy which leads
to the illness. males on the other hand will either survive and be
entirely healthy with a working copy or die, if they are even born if
they have only a defective copy.

\hypertarget{lethal-dominants}{%
\subparagraph{lethal dominants}\label{lethal-dominants}}

lethal dominants may occur in one of two cases. First, if the onset of
the disease/ defect related to the Allele only occurs after reproductive
age. Secondly if a De Novo (random, once off) DNA mutation occurs in a
given individual. lethal dominants can also result from an Allele which
interferes with the formation of aggregates of the protein which the
gene codes for, or receptors of that protein,or finally if it has a
detrimental effect on another protein or gene (such as the yellow mice)

\hypertarget{examples}{%
\paragraph{examples}\label{examples}}

\hypertarget{yellow-mice.}{%
\subparagraph{Yellow Mice.}\label{yellow-mice.}}

\hypertarget{details-4}{%
\subparagraph{details}\label{details-4}}

Y allele codes for the yellow coloured pigment. y codes for other
pigment/no pigment. Y is a lethal Dominant(?) Allele. As a YY individual
with not develop bast the early stages of implantation into the uterine
wall.

\hypertarget{moleculargenetic-explanation}{%
\subparagraph{molecular/genetic
explanation}\label{moleculargenetic-explanation}}

When the Y allele is present at the gene loci it causes a deletion in
which, its own gene promoter, as well as the coding sequence of the gene
upstream of it(the MERC gene necessary in RNA processing) are removed.
the net result being that proteins relating to the gene upstream are no
longer removed, and the pigment gene is now promoter by the MERC
promoter instead of its own promoter

\hypertarget{result-and-phenotype}{%
\subparagraph{Result and Phenotype}\label{result-and-phenotype}}

the result is that Yy individual still produce enough RNA processing
proteins to survive/ avoid miscarriage,but they are more susceptible to
diseases such as obesity and cancer. Also their yellow pigment gene is
promoted heavily during early development leading to yellow coloured
fur. YY miscarriages early in pregnancy as it cant produce necessary RNA
processing proteins. yy is health individual without yellow colour.
birth rations will be skewed by the fact that YY individuals are not
actually born. NOTE: this is an example of incomplete dominance of Y
with respect to MERC.

\end{document}
