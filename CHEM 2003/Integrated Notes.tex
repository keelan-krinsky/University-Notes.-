\PassOptionsToPackage{unicode=true}{hyperref} % options for packages loaded elsewhere
\PassOptionsToPackage{hyphens}{url}
%
\documentclass[]{article}
\usepackage{lmodern}
\usepackage{amssymb,amsmath}
\usepackage{ifxetex,ifluatex}
\usepackage{fixltx2e} % provides \textsubscript
\ifnum 0\ifxetex 1\fi\ifluatex 1\fi=0 % if pdftex
  \usepackage[T1]{fontenc}
  \usepackage[utf8]{inputenc}
  \usepackage{textcomp} % provides euro and other symbols
\else % if luatex or xelatex
  \usepackage{unicode-math}
  \defaultfontfeatures{Ligatures=TeX,Scale=MatchLowercase}
\fi
% use upquote if available, for straight quotes in verbatim environments
\IfFileExists{upquote.sty}{\usepackage{upquote}}{}
% use microtype if available
\IfFileExists{microtype.sty}{%
\usepackage[]{microtype}
\UseMicrotypeSet[protrusion]{basicmath} % disable protrusion for tt fonts
}{}
\IfFileExists{parskip.sty}{%
\usepackage{parskip}
}{% else
\setlength{\parindent}{0pt}
\setlength{\parskip}{6pt plus 2pt minus 1pt}
}
\usepackage{hyperref}
\hypersetup{
            pdfborder={0 0 0},
            breaklinks=true}
\urlstyle{same}  % don't use monospace font for urls
\setlength{\emergencystretch}{3em}  % prevent overfull lines
\providecommand{\tightlist}{%
  \setlength{\itemsep}{0pt}\setlength{\parskip}{0pt}}
\setcounter{secnumdepth}{0}
% Redefines (sub)paragraphs to behave more like sections
\ifx\paragraph\undefined\else
\let\oldparagraph\paragraph
\renewcommand{\paragraph}[1]{\oldparagraph{#1}\mbox{}}
\fi
\ifx\subparagraph\undefined\else
\let\oldsubparagraph\subparagraph
\renewcommand{\subparagraph}[1]{\oldsubparagraph{#1}\mbox{}}
\fi

% set default figure placement to htbp
\makeatletter
\def\fps@figure{htbp}
\makeatother


\date{}

\begin{document}

\hypertarget{tutorial-question-and-additional-work}{%
\section{tutorial question and additional
work}\label{tutorial-question-and-additional-work}}

\hypertarget{chapter-1}{%
\subsection{chapter 1}\label{chapter-1}}

section 1.1-1.3

self test 1.1

\#section one general review

\hypertarget{quantum-numbers}{%
\subsection{quantum numbers}\label{quantum-numbers}}

\hypertarget{principle-quantum-number}{%
\subsubsection{principle quantum
number}\label{principle-quantum-number}}

n

\hypertarget{range}{%
\paragraph{range}\label{range}}

\(n \in mathbb{Z}^+ | n \geq 1\)

\hypertarget{details}{%
\paragraph{details}\label{details}}

determines the enegy level of the shell/ determines the energy level of
the electron.

\hypertarget{orbital-angular-momentum-number.}{%
\subsubsection{orbital angular momentum
number.}\label{orbital-angular-momentum-number.}}

l \#\#\#\# range \(l\in \mathbb{Z} | 0 \leq l \leq n-1\)

\hypertarget{details-1}{%
\paragraph{details}\label{details-1}}

determines the size and shape of the sub shell/ determines the area
around the nucleus which the electron may inhabit.

\hypertarget{magnetic-quantum-number}{%
\subsubsection{magnetic quantum number}\label{magnetic-quantum-number}}

\(m_s\)

\hypertarget{range-1}{%
\paragraph{range}\label{range-1}}

\(m_{l} \in \mathbb{Z} | m_{l}\)

\hypertarget{details-2}{%
\paragraph{details}\label{details-2}}

determines the orientation of the orbital/lobe which the electron is in.

\hypertarget{magnetic-spin-number-m_s}{%
\subsubsection{magnetic spin number
m\_\{s\}}\label{magnetic-spin-number-m_s}}

\hypertarget{range-2}{%
\paragraph{range}\label{range-2}}

\(m_{s} \in \mathbb{Z}\)

\end{document}
